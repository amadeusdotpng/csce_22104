\documentclass[11pt]{report}
\usepackage{textcomp}

\usepackage{titlesec}
\titlespacing*{\section}
{0pt}{\baselineskip}{0em}
\titlespacing*{\subsection}
{0pt}{\baselineskip}{0em}

\usepackage{geometry}
\geometry{left=1in, right=1in, top=1in, textheight=9in}

\usepackage{enumitem}
\newlist{steps}{enumerate}{1}
\setlist[steps, 1]{wide=0pt, leftmargin=\parindent, label=Step \arabic*:}

\usepackage{fancyhdr}
\fancypagestyle{plain}{%
    \fancyhf{} % clear all header and footer fields
    \fancyfoot[C]{\sffamily\fontsize{.75em}{.75em}\selectfont\thepage} % except the center
    \renewcommand{\headrulewidth}{0pt}
    \renewcommand{\footrulewidth}{0pt}
}
\pagestyle{plain}

\usepackage{graphicx}
\graphicspath{ {./media/} }

\usepackage{setspace}
\doublespacing

\usepackage{minted}
\usepackage{xcolor}
\definecolor{LightGray}{gray}{0.9}
% \newmintinline[vhdl]{vhdl}{fontsize=\small, bgcolor=LightGray}

% make fancy title page
\makeatletter
\newcommand{\@labsection}{000}
\newcommand{\labsection}[1]{
    \renewcommand{\@labsection}{#1}
}

\newcommand{\@labnumber}{0}
\newcommand{\labnumber}[1]{
    \renewcommand{\@labnumber}{#1}
}

\newcommand{\@shortsubmitted}{1/1/70}
\newcommand{\shortsubmitted}[1]{
    \renewcommand{\@shortsubmitted}{#1}
}

\lfoot{\footnotesize \textit{University of Arkansas \\ EECS Department}}
\rfoot{\footnotesize \textsl{\@shortsubmitted}}

\renewcommand{\maketitle}{
    \newgeometry{left=1in, right=1in, top=1.75in, textheight=8.25in}
    \singlespacing
    \begin{center}
        {\huge \bf CSCE 22104} \\
        \vspace{2.5em}
        {\Large \bf Lab Report} \\
        \vspace{2em}
        \noindent\rule{20em}{0.4pt} \\
        \vspace{1em}
        {\Large \@author} \\
        \vspace{.75em}
        {\normalsize ID: 011019116} \\
        \vspace{.75em}
        {\normalsize Lab Section \@labsection} \\
        \vspace{.75em}
        {\normalsize Lab \@labnumber} \\
    \end{center}
    \newpage
    \restoregeometry
}

\makeatother


% TEXTWIDTH = 100
\begin{document}
\title{Lab Report 6}
\author{Brent Marcus Orlina}

\labsection{001}
\labnumber{6}

\shortsubmitted{3/19/25}

\maketitle

\section*{Introduction}
This lab's goal was to create a 1-bit ALU component and create a 16-bit ALU component, using 1-bit
components. The ALU components should support four operations: addition, subtraction, logical-and,
and logical-or, using the opcodes \verb|00|, \verb|01|, \verb|10|, and \verb|11|, respectively. The
components should also be asynchronous, i.e. the component does not depend on a clock.

\section*{Approach}
\begin{listing}[h!]
    \inputminted[
        frame=lines,
        breaklines,
        linenos,
        tabsize=4,
        fontsize=\footnotesize,
        bgcolor=LightGray
    ]{vhdl}{./media/ALU1Bit-port.vhd}
    \caption{The 1-bit ALU component's ports.}
    \label{listing:ALU1Bit-port}
\end{listing}

The 1-bit ALU component was first implemented. Listing \ref{listing:ALU1Bit-port} shows the ports
that the ALU takes in. The input port \verb|s| represents the opcode, determining wether the ALU
should execute an addition, subtraction, logical-and, or logical-or operation. The input ports
\verb|a| and \verb|b| are the operands. The input port \verb|cin| is the carry-in from a previous
ALU component, used for the addition and subtraction operations. It is unused when performing a
logical-and or a logical-or operation.

The output port \verb|sout| is the result of the operation that was executed. Output port
\verb|cout| is used for addition and subtraction operations, connected to the next ALU component so
that it can correctly calculate the result. Although \verb|cin|s and \verb|cout|s are not used for
the logical-and and logical-or operations, it will still be calculated regardless. It will not
affect the result since the two logical operations do not use the \verb|cin| input port.

\newpage

\begin{listing}[h!]
    \inputminted[
        frame=lines,
        breaklines,
        linenos,
        tabsize=4,
        fontsize=\footnotesize,
        bgcolor=LightGray
    ]{vhdl}{./media/ALU1Bit_datapath.vhd}
    \caption{The 1-bit ALU component's datapath implementation.}
    \label{listing:ALU1Bit-datapath}
\end{listing}

Listing \ref{listing:ALU1Bit-datapath} shows the implementation for each of the operations that the
ALU supports. Firstly, the signals \verb|sout_adder|, \verb|sout_and|, and \verb|sout_or|,
correspond to the results of each operations that the ALU supports, with \verb|sout_adder|
corresponding to both the addition and subtraction operations. The signal \verb|inverse_b| is used
to support subtraction. The results of all of the operations are calculated with the input port
\verb|s| deciding which result to connect to the output port \verb|sout| by using the operations'
respective opcodes.

Notice that since the result of addition and subtraction operations both correspond to only one
signal. This allows it so that only the $1^{st}$ bit is checked to be a $0$ since both operations'
opcodes have their $1^{st}$ as a $0$. The motivation behind merging both operations into one signal
is that binary subtraction in twos complement is equivalent to
\begin{equation} \label{eq:binary_subtraction}
    A - B = A + (\widetilde{B} + 1)
\end{equation}
therefore, the input port \verb|b| must simply be inverted to support the subtraction operation.
Since the addition and subtraction's opcodes are different in the $0^{th}$ bit where subtraction has
a $1$, \verb|b| can simply be \verb|XOR|ed by the $0^{th}$ bit of the opcode, which inverses
\verb|b| when the opcode signals for a subtraction operation. This potential inverse of \verb|b| is
connected to the \verb|inverse_b| signal, used only by the addition and subtraction operations, so
that it does not affect the other operations in which the $0^{th}$ bit of the opcode is a $1$.

The addition of the extra $1$ to $\widetilde{B}$ in \ref{eq:binary_subtraction} is not implemented
in the 1-bit ALU since if the 16-bit ALU was implemented using sixteen 1-bit ALUs, it would add an
extra 1 in each bit of the 16-bit ALU, producing the wrong result. The extra $1$ will come from the
\verb|cin| of the first 1-bit ALU, again using the $0^{th}$ bit of the opcode, later shown in [REF
to later figure here]. 

The addition operation remains correct as the input port \verb|b| won't be inverted and there
won't be an extra addition of one since the $0^{th}$ bit of the opcode will be $0$. The other two
operations, logical-and and logical-or, are trivial by simply performing an \verb|AND| or an
\verb|OR| between input ports \verb|a| and \verb|b|. The \verb|cout| output port is also somewhat
trivial, implemented similarly to a normal Full Adder. However, it uses the signal \verb|inverse_b|
to support subtraction.


\section*{Experimentation}
Lorem ipsum dolor sit amet, officia excepteur ex fugiat reprehenderit enim labore culpa sint ad nisi
Lorem pariatur mollit ex esse exercitation amet. Nisi anim cupidatat excepteur officia.
Reprehenderit nostrud nostrud ipsum Lorem est aliquip amet voluptate voluptate dolor minim nulla est
proident. Nostrud officia pariatur ut officia. Sit irure elit esse ea nulla sunt ex occaecat
reprehenderit commodo officia dolor Lorem duis laboris cupidatat officia voluptate. Culpa proident
adipisicing id nulla nisi laboris ex in Lorem sunt duis officia eiusmod. Aliqua reprehenderit
commodo ex non excepteur duis sunt velit enim. Voluptate laboris sint cupidatat ullamco ut ea
consectetur et est culpa et culpa duis.

\newpage

\section*{Results \& Discussion}
Lorem ipsum dolor sit amet, officia excepteur ex fugiat reprehenderit enim labore culpa sint ad nisi
Lorem pariatur mollit ex esse exercitation amet. Nisi anim cupidatat excepteur officia.
Reprehenderit nostrud nostrud ipsum Lorem est aliquip amet voluptate voluptate dolor minim nulla est
proident. Nostrud officia pariatur ut officia. Sit irure elit esse ea nulla sunt ex occaecat
reprehenderit commodo officia dolor Lorem duis laboris cupidatat officia voluptate. Culpa proident
adipisicing id nulla nisi laboris ex in Lorem sunt duis officia eiusmod. Aliqua reprehenderit
commodo ex non excepteur duis sunt velit enim. Voluptate laboris sint cupidatat ullamco ut ea
consectetur et est culpa et culpa duis.

Lorem ipsum dolor sit amet, officia excepteur ex fugiat reprehenderit enim labore culpa sint ad nisi
Lorem pariatur mollit ex esse exercitation amet. Nisi anim cupidatat excepteur officia.
Reprehenderit nostrud nostrud ipsum Lorem est aliquip amet voluptate voluptate dolor minim nulla est
proident. Nostrud officia pariatur ut officia. Sit irure elit esse ea nulla sunt ex occaecat
reprehenderit commodo officia dolor Lorem duis laboris cupidatat officia voluptate. Culpa proident
adipisicing id nulla nisi laboris ex in Lorem sunt duis officia eiusmod. Aliqua reprehenderit
commodo ex non excepteur duis sunt velit enim. Voluptate laboris sint cupidatat ullamco ut ea
consectetur et est culpa et culpa duis.

\section*{Conclusions}
Lorem ipsum dolor sit amet, officia excepteur ex fugiat reprehenderit enim labore culpa sint ad nisi
Lorem pariatur mollit ex esse exercitation amet. Nisi anim cupidatat excepteur officia.
Reprehenderit nostrud nostrud ipsum Lorem est aliquip amet voluptate voluptate dolor minim nulla est
proident. Nostrud officia pariatur ut officia. Sit irure elit esse ea nulla sunt ex occaecat
reprehenderit commodo officia dolor Lorem duis laboris cupidatat officia voluptate. Culpa proident
adipisicing id nulla nisi laboris ex in Lorem sunt duis officia eiusmod. Aliqua reprehenderit
commodo ex non excepteur duis sunt velit enim. Voluptate laboris sint cupidatat ullamco ut ea
consectetur et est culpa et culpa duis.

% \newpage
% 
% \section*{References}
% \noindent
% [1]    Computer Organization 22104, EECS, University of Arkansas, “Lab 1,”  Sep. 17, 2024.
% 
% \noindent
% [2]    Computer Organization 22104, EECS, University of Arkansas, “Lab 2,”  Sep. 24, 2024.
% 
% \newpage
% 
% \section*{Appendix}
% \begin{figure}[h!]
%     \centering
%     \includegraphics[width=0.9\textwidth]{foo}
%     \caption{
%         Lorem ipsum dolor sit amet, qui minim labore adipisicing minim sint cillum sint consectetur
%         cupidatat.
%     }
%     \label{fig:foo}
% \end{figure}
% 
% \newpage
% 
% \begin{figure}[h!]
%     \centering
%     \includegraphics[height=0.4\textheight]{bar}
%     \caption{Lorem ipsum something something shorter sentence}
%     \label{fig:bar}
% \end{figure}
\end{document}
