\documentclass[11pt]{report}
\usepackage{textcomp}

\usepackage{titlesec}
\titlespacing*{\section}
{0pt}{\baselineskip}{0em}
\titlespacing*{\subsection}
{0pt}{\baselineskip}{0em}

\usepackage{geometry}
\geometry{left=1in, right=1in, top=1in, textheight=9in}

\usepackage{enumitem}
\newlist{steps}{enumerate}{1}
\setlist[steps, 1]{wide=0pt, leftmargin=\parindent, label=Step \arabic*:}

\usepackage{fancyhdr}
\fancypagestyle{plain}{%
    \fancyhf{} % clear all header and footer fields
    \fancyfoot[C]{\sffamily\fontsize{.75em}{.75em}\selectfont\thepage} % except the center
    \renewcommand{\headrulewidth}{0pt}
    \renewcommand{\footrulewidth}{0pt}
}
\pagestyle{plain}

\usepackage{graphicx}
\graphicspath{ {./media/} }

\usepackage{setspace}
\doublespacing

\usepackage{minted}
\usepackage{xcolor}
\definecolor{inlinebg}{rgb}{0.95, 0.95, 0.95}
\newmintinline[vhdl]{vhdl}{fontsize=\small, bgcolor=inlinebg}

% make fancy title page
\makeatletter
\newcommand{\@labsection}{000}
\newcommand{\labsection}[1]{
    \renewcommand{\@labsection}{#1}
}

\newcommand{\@labnumber}{0}
\newcommand{\labnumber}[1]{
    \renewcommand{\@labnumber}{#1}
}

\newcommand{\@shortsubmitted}{1/1/70}
\newcommand{\shortsubmitted}[1]{
    \renewcommand{\@shortsubmitted}{#1}
}

\renewcommand{\maketitle}{
    \newgeometry{left=1in, right=1in, top=1.75in, textheight=7.5in}
    \singlespacing
    \begin{center}
        {\huge \bf CSCE 22104} \\
        \vspace{2.5em}
        {\Large \bf Lab Report} \\
        \vspace{2em}
        \noindent\rule{20em}{0.4pt} \\
        \vspace{1em}
        {\Large \@author} \\
        \vspace{.75em}
        {\normalsize ID: 011019116} \\
        \vspace{.75em}
        {\normalsize Lab Section \@labsection} \\
        \vspace{.75em}
        {\normalsize Lab \@labnumber} \\

        
    \end{center}
    \lfoot{\footnotesize \textit{University of Arkansas \\ EECS Department}}
    \rfoot{\footnotesize \textsl{\@shortsubmitted}}
    \newpage
    \restoregeometry
}

\makeatother


% TEXTWIDTH = 100
\begin{document}
\title{Lab Report 1}
\author{Brent Marcus Orlina}

\labsection{001}
\labnumber{1}

\shortsubmitted{2/5/25}

\maketitle

\section*{Introduction}
This lab's goal was to create a component, ``10-Counter,'' in VHDL that counts from 0 to 9 that
increments during the rising edge of a clock cycle. The counter also loops back to 0 after
reaching 9. The second goal of this lab was to create a clock component that displays the hours,
minutes, and seconds and increments by one second during the rising edge of a clock cycle. The clock
also loops back to displaying all zeroes after reaching 24 hours.

\section*{Approach}
For the 10-Counter component, there are three 1-bit input ports, \verb|clk|, \verb|rst|, and
\verb|en| and a 4-bit output port, \verb|out|. When \verb|clk| is in a rising edge and \verb|en| is
active, \verb|out| is incremented. However, when \verb|rst| is active, it overrides \verb|en| and
synchronously resets \verb|out| to 0.

To implement this a process sensitive to all three input ports is used. \verb|clk| is checked if it
is in a rising edge with \vhdl{clk'event and clk = '1'}. If the condition is satisfied, \verb|rst|
is checked if it is active with \vhdl{rst = '1'}. If it is active, a 4-bit signal \verb|count| is
set to 0, with \vhdl{count <= "0000"}. Otherwise, \verb|en| is checked if it is active. If it is,
the signal \verb|count| is set back to 0 if it reaches 10, otherwise it is incremented with
\vhdl{count <= count + 1}. The library \verb|IEEE.std_logic_unsigned.all| is required to be able to
increment \verb|count| by using addition. Outside of the process, the signal \verb|count| is
connected to \verb|out| with \vhdl{out <= count}. The signal \verb|count| is necessary since it is
illegal to set an output port to a value that uses that same output port in VHDL.

For the clock component, there are three 1-bit input ports, \verb|clk|, \verb|rst|, and \verb|en|, 
one 1-bit output port \verb|cout|, one 4-bit output port \verb|H| and two 5-bit output ports
\verb|M| and \verb|S|. Similar to the 10-Counter, when \verb|clk| is in a rising edge and \verb|en|
is active, the output port \verb|S| is incremented. When \verb|S| reaches 60, it is set back to 0
and the output port \verb|M| is incremented. Similarly, when \verb|M| reaches 60, it is set back to
0 and the output port \verb|H| is incremented. Finally, when \verb|H| reaches 24, it is set back to
0. \verb|cout| is active when \verb|H| is incremented to 23.

To implement this, two more components are made, ``60-Counter'' and ``24-Counter.'' These counters
use ``10-Counter'' as the base. The two other counters instead count up to 60 and 24 respectively.
Another difference is that these counters have an extra 1-bit output port \verb|cout|. These are
used to chain the counters together. In the clock, the 60-Counter responsible for \verb|S| activates
its \verb|cout| when it would be incremented to 59. The \verb|cout| is then connected to the
60-Counter responsible for \verb|M|'s \verb|en| input port. This is so that in the next cycle, when
\verb|S| is incremented to 60 and set back to 0, \verb|M| is incremented, since there are 60 seconds
in a minute. This is also the same for the \verb|M|'s 60-Counter and \verb|H|'s 24-Counter.

In the clock component two instances of 60-Counter and one instance of 24-Counter is created. The
clock's \verb|clk| and \verb|rst| input ports are connected to each of the counter's \verb|clk| and
\verb|rst| input ports. The clock's \verb|en| is connected to the first 60-Counter's \verb|en|. The
first 60-Counter's \verb|out| is connected to \verb|S| and its \verb|cout| is connected to the
second 60-Counter's \verb|en| using a signal \verb|sec_carry|. The second 60-counter's \verb|cout|
is connected to a signal \verb|min_carry| and its \verb|out| is connected to \verb|M|. The signals
\verb|sec_carry| and \verb|min_carry| are connected to the 24-Counter's \verb|en| through an AND
gate. This is because an hour should only increment once both the second and minute gets incremented
to 60. If only the \verb|min_carry| signal was connected, it would increment the 24-Counter for 60
cycles when the second 60-Counter is incremented to 59 producing the wrong behavior.

\section*{Experimentation}
The components' behaviors were tested using hand-crafted testbenches. The testbench for the
10-Counter tested all possible combinations of the ports \verb|en| and \verb|rst| signals as well as
testing the behavior of the counter incrementing from 9. The counter worked as expected and
incremented during the rising-edge of every clock cycle when \verb|en| was active. The counter
looped back to 0 once it incremented from 9. The reset behavior worked as expected with the counter
resetting back to 0 when \verb|rst| was active and taking priority over \verb|en|.

The testbench for the clock component tested all possible combinations of the \verb|en| and
\verb|rst| signals. It also tested the behavior of ports \verb|S| and \verb|M| incrementing during
the correct times as well as looping back to 0 when they've reached their defined maximum limit. The
behavior of port \verb|H| was only tested for incrementing during the correct times. The component
worked as expected and \verb|S| incremented during the rising edge of every clock cycle when
\verb|en| was active. The port \verb|M| correctly incremented only when \verb|S| was incremented
from 59 and looped back to 0. \verb|H| correctly incremented only when \verb|S| and \verb|M| were
both incremented from 59 and looped back to 0. Similarly, the reset behavior worked as expected with
\verb|S|, \verb|M|, and \verb|H| resetting back to 0 when \verb|rst| was active and taking priority
over \verb|en|.

\newpage

\section*{Results \& Discussion}
\begin{figure}[h!]
    \centering
    \includegraphics[width=0.95\textwidth]{10-counter-waveform}
    \caption{
        The waveform for the 10-Counter component implementation.
    }
    \label{fig:10-counter-waveform}
\end{figure}

The 10-Counter works as expected as shown in figure \ref{fig:10-counter-waveform}. The counter has
to first be reset, by activating signal \verb|Reset|, so that the output is initialized to 0. The
signal \verb|Enable| is then enabled. It shows that the signal \verb|Output| increments during the
rising edge of every clock cycle when \verb|Enable| is active. It also shows that the counter resets
back to 0 it would increment to 10 in the 115th nanosecond. In the 150th nanosecond, \verb|Enable|
deactivates and the counter correctly stops incrementing. In the 175th nanosecond, it shows that
when \verb|Reset| is active, it synchronously resets the output back to 0. It can also be seen in
the 215th nanosecond that when both \verb|rst| and \verb|en| are both active, \verb|rst| takes
priority and sets \verb|output| to 0. 

The testbench for this component is sufficient since it tests all the possible combinations of
\verb|en| and \verb|rst|. It also checks the behavior of the clock when it is incremented from 9,
with the expected behavior that it loops back to 0.

\begin{figure}[h!]
    \centering
    \includegraphics[width=0.90\textwidth]{clock-waveform_en_rst}
    \caption{
        Segment of the waveform for the clock component implementation. Shows the enable and reset
        functionality.
    }
    \label{fig:clock-waveform_en_rst}
\end{figure}

The clock component also works as expected. Figure \ref{fig:clock-waveform_en_rst} shows the reset
and enable functionality of the clock component. In the 36,915th nanosecond, the signal on the
testbench \verb|Reset| activates and the clock correctly and synchronously resets all of its output
ports back to 0. In the 36,970th nanosecond, the signal on the testbench \verb|Enable| deactivates
and the clock correctly stops incrementing. In the 37,045th nanosecond, \verb|Reset| activates while
\verb|Enable| is also active, and the clock resets back to 0, correctly taking priority over
\verb|Enable|.

\begin{figure}[h!]
    \centering
    \includegraphics[width=0.90\textwidth]{clock-waveform_second_tick}
    \caption{
        Segment of the waveform for the clock component implementation. Shows how seconds part
        increments.
    }
    \label{fig:clock-waveform_second_tick}
\end{figure}

\begin{figure}[h!]
    \centering
    \includegraphics[width=0.90\textwidth]{clock-waveform_minute_tick}
    \caption{
        Segment of the waveform for the clock component implementation. Shows how minutes part
        increments.
    }
    \label{fig:clock-waveform_minute_tick}
\end{figure}

\begin{figure}[h!]
    \centering
    \includegraphics[width=0.90\textwidth]{clock-waveform_hour_tick}
    \caption{
        Segment of the waveform for the clock component implementation. Shows how hours part
        increments.
    }
    \label{fig:clock-waveform_hour_tick}
\end{figure}

In figure \ref{fig:clock-waveform_second_tick}, the seconds increments during the rising edge of
every cycle when \verb|Enable| is on, which is the correct behavior. It also shows that the clock is
set back to 0 when \verb|rst| is active. Figure \ref{fig:clock-waveform_minute_tick} shows that the
\verb|Minutes| increments from 0 to 1 when the seconds is incremented from 59 back to 0. This is the
correct behavior since there are 60 seconds in 1 minute. Similarly, figure
\ref{fig:clock-waveform_hour_tick} shows that the \verb|Hours| increments from 0 to 1 when both the
\verb|Minutes| and \verb|Seconds| increment from 59 back to 0 since there are 60 minutes in 1 hour.

The testbench for the clock component is not completely sufficient. It tests the \verb|Reset| and
\verb|Enable| functionality. It tests for the behaviors for when \verb|Seconds|, \verb|Minutes|, and
\verb|Hours| should increment. It also tests for the behavior of when \verb|Seconds| and
\verb|Minutes| increments from 59. However, \verb|Hours| incrementing from 23 and the output for
\verb|Midnight|, which should be active when \verb|Hours| is incremented to 23, was not verified.
This was because the software used was not able to simulate the clock cycling all the way to 24+
hours worth of clock cycles.

\section*{Conclusions}
Both the 10-Counter component and clock component displayed the correct behavior during testing. The
knowledge learned from this lab was learning counters worked and how they can be constructed as well
as chaining different counters together. The skills practiced in this lab was learning how to build
components behaviorally and structurally, and creating sufficient testbenches that test the
components behaviors with different inputs. Debugging VHDL code was also an important skill
practiced.

% \newpage
% 
% \section*{References}
% \noindent
% [1]    Computer Organization 22104, EECS, University of Arkansas, “Lab 1,”  Sep. 17, 2024.
% 
% \noindent
% [2]    Computer Organization 22104, EECS, University of Arkansas, “Lab 2,”  Sep. 24, 2024.
% 
% \newpage
% 
% \section*{Appendix}
% \begin{figure}[h!]
%     \centering
%     \includegraphics[width=0.9\textwidth]{foo}
%     \caption{
%         Lorem ipsum dolor sit amet, qui minim labore adipisicing minim sint cillum sint consectetur
%         cupidatat.
%     }
%     \label{fig:foo}
% \end{figure}
% 
% \newpage
% 
% \begin{figure}[h!]
%     \centering
%     \includegraphics[height=0.4\textheight]{bar}
%     \caption{Lorem ipsum something something shorter sentence}
%     \label{fig:bar}
% \end{figure}
\end{document}
